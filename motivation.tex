
While whole-brain level simulators have been developed and published for
several years now, making the final step of connecting these simulations to
empirical results has remained a challenge due to several factors:

\begin{enumerate}

	\item Source code is typically not distributed, effectively closing
	the behavior, black box, etc. 

	\item The forward solutions required to obtain simulated M/EEG \& fMRI
	data are non trivial, requiring interaction with several pieces of software

	\item Published simulation methods for stochastic, delayed systems on
		surfaces are almost non existent (XPPAUT is a notable exception).
		Efficient handling of $N^2$ delays requires custom routines.

	\item Managing all of the different computational pieces is typically
	challenging for those who work with empirical data.

\end{enumerate}

To address these concerns, a flexible architecture was developed to
allow easy integration of any computational tools along with a system
for describing typically types of data. A web based UI was developed
for users not comfortable with programming, as well as MATLAB toolbox
for interacting with the Python based framework, given that many
neuroscientists are already comfortable with the MATLAB workflow.
Lastly, a high performance, highly documented simulator along with
various forward solutions have been implemented and released under a
GPL licence to ensure universal access to high quality simulations, 
developed on the well-known Github, making it extremely easy for 
anyone to contribute.


Large scale simulation implies flexible integration. We shall see
how this is enable by the architecture..
\note[mw]{expand}

In effect, we wish for a theoretician and clinician to be able to
collaborate; to enable such a possibility, we require a platform
to enable such opportunity. 
\note[mw]{expand}


\subsection{Why another simulator?}

A significant part of \TVB is simulating brain-scale neural networks. While
several existing simulators could have been adapted, we have estimated that
\TVB style simulations are far enough outside the design of other simulators to
make starting from scratch a better idea.

Many neural network simulators have been developed and published, focusing
first on abstract rate neurons (in the style of PDP), modelling neurocognitive
processes, on one hand, and on the other, full multicompartmental neuron
simulators treating complex spatial geometries, e.g. NEURON \cite{Hines_2001}.
More recently, due to interest in the computational properties of spiking 
neurons and their relevance to experimental observations, simulators targeting
specifically spiking neurons have been prominent, e.g. Brian 
\cite{Goodman_2009}.

However, another level description of neural dynamics has been treated
in the literature of neural mass models and neural fields 
\cite{Deco_2008a, Coombes_2010}. Here, the spatial
extent of the modeled dynamics is far larger and hence permits networks 
thereof to scale reasonably to the entire cortex, under the assumptions 
of the models, when combined with empirical measurements of cortico-
cortical connectivity. Therefore, the physical scale modeled by the \TVB
simulators differs from that for which other simulators were designed.
Several technical issues stem from this scale, e.g. efficient handling
of dense $N^2$ delays and neural field-like connectivity, which will be
discussed in more detail below. 

\subsection{Why Python?}

In fact, the core simulator began in MATLAB, however, as the needs 
expanded, the architecture, detailed in the next section, quickly 
outgrew the matrix-struct-function triumvirate that is conventional
in MATLAB programming. While modern MATLAB permits advanced object-oriented
programming, it has the disadvantages of being relatively unused, and
largely unsupported by MATLAB's own IDE, the MATLAB Compiler, and the free
alternative Octave, lastly it provides no support for metaclasses or data
descriptors, which were extensively used in \TVB's architecture.

The architecture of \TVB had been prototyped in Python, and in turn, both the
language and the scientific ecosystem were more than rich enough to support
continued developed entirely within Python, of both the architecutre and the
simulator, in addition to it being a general purpose language. 
\note[sk]{Perhaps: The benifits gained, when dealing with contributors of, 
shall we say, varying backgrounds, by Python's strong focus on style 
conventions.}

