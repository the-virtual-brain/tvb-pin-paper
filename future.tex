
Since the recent release of the $1.0$ version of TVB, it has been 
officially considered \textit{feature} complete, however, in several
cases, the development of features has outstripped other essential 
parts of software projects. Going forward, general priorities include
advancing test coverage, improving documentation for users, and
preparing PyPI and Debian packages. In the mean
time, TVB's Google groups mailing list continues to fill any gaps. 

In the simulator itself, continued optimization of C and GPU code generation
will take place to increase the rate at which parameter sweeps can be
performed. Additionally, an interface \textit{from} MATLAB to TVB 
is being developed to allow use of the simulator through a simple
set of MATLAB functions. As this infrastructure is based on an HTTP and 
JSON API, it will likely enable other applications to work with TVB as well.

Lastly, as TVB was originally motivated to allow a user to move from
acquired data to simulated data as easily as possible, we will continue
to integrate several of the requisite steps that are not currently 
covered, such as analysis of DSI data to produce connectivity matrices,
though in many cases, such as parcellation and labeling of 
cortical areas, these steps may continue to require 
intervention with other software. Nevertheless, TVB provides a promising 
platform for integrating 
neuroinformatics tools with an emphasis on analysis and modeling of 
human neuroimaging data.

%\begin{enumerate}
%	\item Diffusion tensor imaging \& tractography pipeline to convert
%        MRI data into connectivity data.
%	\item Connectome project on structural connectivity
%	\item Structural imaging processing via FreeSurfer (pySurfer, NiPy, etc.)
%	\item NeuroML model specification 
%\end{enumerate}

