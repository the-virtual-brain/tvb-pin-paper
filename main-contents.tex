%!TEX TS-program = pdflatex                                                    %
%!TEX encoding = UTF8                                                          %
%!TEX spellcheck = en-US                                                       %
%------------------------------------------------------------------------------%
% to compile use "latexmk --pdf main.tex"                                      %
%------------------------------------------------------------------------------%
% to count words 
% "pdftotext main_nofigs_nocaptions.pdf - | egrep -e '\w\w\w+' | iconv -f ISO-8859-15 -t UTF-8 | wc -w"
% -----------------------------------------------------------------------------%

\usepackage{url}
\usepackage[english]{babel}
\usepackage[utf8]{inputenc}
\usepackage[T1]{fontenc}
%\usepackage[pdftex]{graphicx} 
%\usepackage{graphics}
%\usepackage{hyperref}
\usepackage{float}
\floatplacement{figure}{H}
\usepackage{booktabs}     % nice tables
\usepackage{tabularx}     % even nicer tabular environments 
\usepackage{amsmath}
\usepackage{amsfonts}
\usepackage{amssymb}
%\usepackage{multicol}
\usepackage{listings}
\usepackage{tikz,times}
\usepackage{courier}
\usepackage[scaled]{beramono}

\usepackage{fancyvrb}

\usetikzlibrary{shapes,arrows}
\usetikzlibrary{arrows,positioning}
\usepackage{xcolor}
\usepackage[font=bf]{subfig}
\usepackage[inline]{trackchanges}

%%%%%%% setup track changes "editors"

\addeditor{mw}
\addeditor{lp}
\addeditor{psl}
\addeditor{ld}
\addeditor{sk}
\addeditor{vj}
\addeditor{jm}

%\usepackage{sectsty}
%\sectionfont{\normalsize\bfseries}
%\usepackage[labelfont=bf]{caption}
%\usepackage{endfloat} %place figures at end of document
%------------------------------------------------------------------------------%
%\captionsetup{
%%format = hang,                % caption format
%labelformat = simple,          % caption label : name and number
%labelsep = period,             % separation between label and text
%textformat = simple,           % caption text as it is
%justification = justified,     % caption text justified
%singlelinecheck = true,        % for single line caption text is centered
%font = {up,singlespacing},     % defines caption (label & text) font
%labelfont = {bf,footnotesize}, % NOTE: tiny size is not working
%textfont = footnotesize,
%%width = \textwidth,           % define width of the caption text
%skip = 1ex,                    % skip the space between float and caption
%listformat = simple,           % in the list of floats, label + caption
%}

%------------------------------------------------------------------------------%
%\hypersetup{
%    bookmarks=true,         % show bookmarks bar?
%    unicode=false,          % non-Latin characters in Acrobat’s bookmarks
%    pdftoolbar=true,        % show Acrobat’s toolbar?
%    pdfmenubar=true,        % show Acrobat’s menu?
%    pdffitwindow=false,     % window fit to page when opened
%    pdfstartview={FitH},    % fits the width of the page to the window
%    pdftitle={TheVirtualBain},    % title
%    pdfauthor={PSL},        % author
%    pdfsubject={ProposedArticle},   % subject of the document
%    pdfcreator={paupau},    % creator of the document
%    pdfnewwindow=true,      % links in new window
%    colorlinks=true,       % false: boxed links; true: colored links
%    linkcolor=red,          % color of internal links (change box color with linkbordercolor)
%    citecolor=blue,        % color of links to bibliography
%    filecolor=magenta,      % color of file links
%    urlcolor=blue           % color of external links
%}
%-----------------------------------------------------------------------
%\usepackage{subcaption}

%%%%%%%%%%%%%%%%%%%%%%%%%%%%%%%%%%%%%%%%%%%%%%%%%%%%%%%%%%%%%%%%%%%%%%%%%%%%%%%%
%%                             New and renew commands                         %%
%%%%%%%%%%%%%%%%%%%%%%%%%%%%%%%%%%%%%%%%%%%%%%%%%%%%%%%%%%%%%%%%%%%%%%%%%%%%%%%%
  
\renewcommand{\lstlistingname}{Code}
\renewcommand{\thesubfigure}{\Alph{subfigure}}
%\newcommand{\inputTikZ}[2]{\scalebox{#1}{\input{#2}}}
\newcommand*{\h}{\hspace{5pt}}   % for indentation
\newcommand*{\hh}{\h\h}          % double indentation
\newcommand*{\tvbmodule}[1]{{\textsc{#1}}}          % scientific modules in "simulator"
\newcommand*{\tvbdatatype}[1]{\textbf{\emph{#1}}}   % datatypes in "datatypes"
\newcommand*{\tvbclass}[1]{{\ttfamily\emph{#1}}}    % classes either in simulator mods or datatypes
\newcommand*{\tvbmethod}[1]{{\textsf{#1}}}          % methods
\newcommand*{\tvbattribute}[1]{{\ttfamily{#1}}}     % attributes
\newcommand*{\tvbtrait}[1]{{\ttfamily{#1}}}         % traited types
\newcommand{\TVB}{\textit{TheVirtualBrain }}

\newcommand{\matlab}{MATLAB}

%%%%%%%%%%%%%%%%%%%%%%%%%%%%%%%%%%%%%%%%%%%%%%%%%%%%%%%%%%%%%%%%%%%%%%%%%%%%%%%%
%%                            Colors and graphics                             %%
%%%%%%%%%%%%%%%%%%%%%%%%%%%%%%%%%%%%%%%%%%%%%%%%%%%%%%%%%%%%%%%%%%%%%%%%%%%%%%%%
\definecolor{palegreen}{HTML}{DAFFDA}
\definecolor{lightgray}{rgb}{0.15,0.15,0.15}
\definecolor{orange}{HTML}{FF7300}
\DeclareGraphicsExtensions{.jpg,.pdf,.png,.tiff}%,.mps,.bmp
\graphicspath{{figures/}}
 
%##--------------------------------------------------------------------------##%
%##                               START HERE                                 ##%
%##--------------------------------------------------------------------------##%
\copyrightyear{}
\pubyear{}

\begin{document}
\lstset{language=Python,
	captionpos=b,
	keepspaces=true,
	numbers=none,
	showspaces=false,
    float=*,
	basicstyle=\fontsize{8pt}{8}\ttfamily
	} 

\firstpage{1}

%%  Authorship and Title
\title[TVB]{Integrating neuroinformatics tools in TheVirtualBrain}
\author[Woodman {et~al}]{
        M. Marmaduke Woodman\,$^{1,*}$,  
        Laurent Pezard\,$^{1}$,  
        Lia Domide\,$^{3}$, 
        Stuart Knock\,$^{1}$, 
        Paula Sanz Leon\,$^{1}$, 
        Jochen Mersmann\,$^{2}$,
        Anthony R. McIntosh \,$^{4}$ and  
        Viktor Jirsa\,$^{1}$\footnote{to whom correspondence should be addressed: marmaduke.woodman@univ-amu.fr,
        viktor.jirsa@univ-amu.fr}}

\address{$^{1}$ Institut de Neurosciences des Syst{\`e}mes, Aix-Marseille
    Universit\'e,  27, Bd. Jean Moulin, 13005, Marseille, France.\\
         $^{3}$ Codemart, 13, Petofi Sandor, 400610, Cluj-Napoca, Romania.\\
         $^{2}$ CodeBox GmbH, Hugo Eckener Str. 7, 70184 Stuttgart, Germany.\\
         $^{4}$ Rotman Research Institute at Baycrest, Toronto, M6A 2E1, Ontario, Canada\\
        }

\history{}

\editor{}

\maketitle

%##--------------------------------------------------------------------------##%
%##                               ABSTRACT                                   ##%
%##--------------------------------------------------------------------------##%


\begin{abstract}
\section{}

TheVirtualBrain (TVB) is a neuroinformatics Python package representing the
convergence of clinical, systems, theoretical neuroscience in the integration,
analysis, visualization and modeling of neural dynamics of the human brain as
well as the imaging modalities through which these dynamics are measured.
TVB is composed of a flexible simulator for both neural dynamics
and modalities such as electroencephalography (EEG), magnetoencephalography
(MEG) and functional magnetic resonance imaging (fMRI), common analysis
techniques such as wavelet decomposition and multiscale sample entropy,
interactive visualizers for replaying cortical time series on the 3D surface or
editing large-scale connectivity matrices, all accessible via an 
web browser user interface.
A datatype system modeling neuroscientific data ties together these pieces with
persistent data storage, based on a combination of SQL \& HDF5.
 These datatypes combine with adapters
allowing TVB to integrate other algorithms or computational systems,
such as MATLAB, for which support is already provided.
TVB provides infrastructure for multiple projects and
multiple users, possibly participating under multiple roles. For example, a
clinician might import patient data to first identify, based on 
electrophysiological dynamics, several potential lesion point on the patient's 
connectome, as obtained from diffusion spectrum imaging (DSI). 
These lesion points are picked up by a modeler, working on the same project, 
and tested for viability through whole brain simulation, based on the 
patient's connectome, and subsequent analysis of dynamical features.
This workflow is one of several multi-user use cases for which TVB conceived.
TVB also drives research forward: the
simulator itself represents the culmination of several recent yet ad-hoc
simulations in the modeling literature on human resting state. The 
availability of the numerical methods, set of neural mass models and 
forward solutions allows for the construction of a wide range of 
brain-scale simulation scenarios.
TVB is therefore a platform for various tools and disparate expertises, supporting the
analysis and modeling of structural and function data from the human brain.  
We will briefly outline the history and motivation for TVB, describing the 
framework and simulator, giving usage examples in the web UI
and Python scripting.
  
\section{Keywords:} large-scale brain network, simulation,  web platform, Python, virtual
brain, connectivity, connectome, neural mass, time delays

\end{abstract}

%\tableofcontents

\section{Introduction}
The advancement of neuroscience and more generally brain and behavioral sciences require
significant interaction between scientific disciplines in order to fulfill their goal
of understanding the relationship between brain and cognition\footnote{Several
important scientific projects such as The Human Brain Project are clear
illustration of this.}. Nevertheless, one major drawback of this
interdisciplinary enterprise is the necessary distribution of competences, most
frequently between individuals but also, in a non negligible number of cases,
between institutes.  Moreover, the technicalities involved in data
analysis and brain simulation usually prevents an optimal diffusion of these
advances to the more experimentally oriented part of the community. 

These problems invite at least two approaches to their solution. Firstly,  the
development of up to date and accessible software libraries developed in commonly
used programming languages and secondly, the development of tools for sharing
competences and data. Due to the high pace of new developments, these solutions
should also remain open to incoming new tools. Several projects fall in the
first category (from SPM to connectivity toolbox, fieldtrip in the \matlab{} ``galaxy''
and from ``nitime'' to neo... to name only a few). In the second category CARMEN
(http://www.carmen.org.uk/) and G-Node (https://portal.g-node.org/data/)
\note[lp]{Other projects?} are (web) platforms for collaborative work and data
sharing. The diffusion of \matlab{} and its integrated environment also usually
take place of the platform / development environment. 
\note[sk]{not exactly sure what this last sentence is trying to say, so, needs rewording}

TheVirtualBrain (TVB) provides its own solutions to these two problems.  In an
initial step, these two problems were addressed in two independent projects: one
whole brain-simulator library developed in \matlab{} and a web platform for
collaborative interactions in the context of multi-purpose data analysis which
was developed in Python.  In each case, the choice of the language was dictated
both by the scientific context and the technical constraints. The whole brain
simulation library was developed in \matlab{} due to the \note[lp]{unfortunate}
widespread use of the language in the neuroscientific community. On the
other hand, the platform was designed for a more generic purpose, such as providing several
interfaces for user interaction (web and console) and the capability to adapt
to existing and future libraries. In this latter case, the programming language
should adapt to tasks from  web development to numerical methods implementation
and should be able to act as a ``glue'' language. These constraints
\note[lp]{naturally} oriented the development of the framework toward
Python\footnote{The first issue of Python in Neurosciences also confirmed the
choice of the Python language.}.  Moreover, Python is a language that provides
the main blocks needed for the project: database interactions, web programming,
object-oriented programming and abstraction. \emph{TheVirtualBrain} benefits
from this initial development which remain in the overall organization of TVB's
architecture: the \emph{framework} and the \emph{scientific library}.

\note[sk]{Terminology (consistency with usage in code naming, docs, previous publications):
          Architecture => overall design (think whitepaper and Lia's diagrams of TVB's architecture;
          Framework => the specific component/package covering database back-end, 
            web interface, etc...)}

\subsection{Why another project? (TVB compared to others)}

The reasons for developing a new project are different for each component of
TVB.

\subsubsection{The framework}
\note[sk]{this subsubsec repeats itself a bit, rearrange existing into 2 paras,
ideally incorporating any comment Lia makes re lp's note}

In effect, we wish for a theoretician and a more experimentally-oriented
neuroscientist to be able to collaborate; to enable such a possibility, we
require a platform that simultaneously provides power, flexibility, and a 
simple interface.
To address these concerns, a flexible architecture was developed to
allow easy integration of any computational tools along with a system
for describing typical types of data. A web based UI was developed
for users not comfortable with programming.
% as well as \matlab{} toolbox for interacting with the Python based framework,
% given that many neuroscientists are already comfortable with the \matlab{}
% workflow.

The two main constraints for the architecture were then to provide a web
interface to allow remote collaboration and a data exchange system to allow the
exchange of data (experimental or simulated) between scientists. To this end,
the particular design chosen for TVB's architecture was the well-known
``model-view-controller'' pattern \note[lp]{detail this}. Initially, an attempt 
was made to develop the architecture using an existing integrated / high-level web
framework, but we rapidly found ourselves struggling with the implementation, using
a framework that did not fit our needs. Due to the availability of a wide range of
Python Packages, we then choose to use more specific blocks for each purpose and
build the necessary interaction between them ourselves. For this purpose, we choose
\textsf{CherryPy} for the web  and \textsf{SQLAlchemy} for the database
exchanges\footnote{Other dependencies of TVB are listed in
    \texttt{TVB\_INSTALL\_REQUIREMENTS} which currently lists
    \texttt{"apscheduler", "beautifulsoup", "cherrypy", "cfflib", "formencode",
        "lxml", "minixsv", "mod\_pywebsocket", "networkx", "nibabel", "numpy",
        "numexpr", "psutil", "scikit-learn", "scipy", "simplejson",
        "sqlalchemy", "sqlalchemy-migrate"}.}

In addition to this, to fulfill the requirement of 3D data visualization, we 
made use of the availability of WebGL in modern browsers.

The architecture of TVB had been prototyped in Python, and in turn, both the
language and the scientific ecosystem were more than rich enough to support
continued developed entirely within Python, of both the framework and the
simulator, in addition to it being a general purpose language. Lastly, Python's
emphasis on readability and idiomatic style facilitates integration of 
code contributions from programmers with disparate backgrounds.

\note[lp]{For Lia: Was Python a "good" choice? and why? Which other language would have
done the job?}

\subsubsection{The simulator}

In the case of the simulator, the situation was a little different, 
a set of \matlab{} routines had been developed that did the job but were never
intended for generic deployment. The needs of the framework exceeded the 
matrix-struct-function triumvirate that is conventional in \matlab{} 
programming. While modern \matlab{} permits advanced object-oriented 
programming, it has the disadvantages of being relatively unused, and 
largely unsupported by \matlab{}'s own IDE, the \matlab{} Compiler, and the 
free alternative Octave, lastly it provides no support for metaclasses or 
data descriptors, which were extensively used in TVB's architecture. The choice
was between: develop our own simulator by porting the existing \matlab{} 
routines to Python and a more generic structure, or make use of an existing
simulation library. 

As in the case of the framework, we started by trying to use an existing 
simulator, namely Brian \cite{Goodman_2009} since it has a very generic way of
specifying models by differential equation expressions. Although we had some 
success in the implementation of several models in Brian we found ourselves in
a similar situation to that with the framework, having to hack around 
incompatibilities between the purpose for which Brian was designed and our 
goals. The major constraint came from the scale of the simulation in 
TVB, which is clearly different from the more usual ``cellular'' simulations.
In addition to requiring support for neural-mass and neural-field models, the
nature of the large-scale connectivity means the simulator needs to efficiently
support a large number of distinct delays. Furthermore, the simulator should be
able to produce simulated experimental data at the macroscopic level, such as 
EEG, MEEG and fMRI, which are not usually considered in the current neuronal 
simulators. Despite these differences, there remains much commonality between 
design considerations for TVB's simulator and the overall structure of neuron 
level simulators. This enabled us to borrow concepts such as \emph{monitors}
(which are objects that record the course of a simulation) from Brian and modify
them to the context of TVB's simulations.


While whole-brain level simulators have been developed and published for
several years now, making the final step of connecting these simulations to
empirical results has remained a challenge due to several factors:

\begin{enumerate}

	\item Source code is typically not distributed, effectively closing
        the behavior, black box, etc. \note[lp]{Is this really true for NEURON,
        GENESIS and others? I doubt...} \note[sk]{I think this item is valid, 
        as it is in relation to ``whole-brain level simulators'', which 
        NEURON and GENESIS are not.}

	\item The forward solutions required to obtain simulated M/EEG \& fMRI
	data are non trivial, requiring interaction with several pieces of software

	\item Published simulation methods for stochastic, delayed systems 
		are almost non existent (XPPAUT \cite{XPPAUT} is a notable exception).
		Efficient handling of $N^2$ delays requires custom routines.

	\item Managing all of the different computational pieces is typically
	challenging for those who work with empirical data.

\end{enumerate}

\note[sk]{This para should probably be dropped, as lp's contribution, which I've
edited above, covers the same content in more detail...}
A significant part of TVB is simulating brain-scale neural networks. While
several existing simulators could have been adapted, we have estimated that
TVB style simulations are far enough outside the design of other simulators to
make starting from scratch a better idea.

\note[sk]{These next two paras provide, primarily, scientific motivation,
whereas the current first two paras provide Python and ``from-scratch''
motivation. I'd be inclined to move the scientific stuff upfront, which will
require minor rewording to make it flow...}
Many neural network simulators have been developed and published, focusing
first on abstract rate neurons (in the style of PDP), modeling neurocognitive
processes, on one hand, and on the other, full multicompartmental neuron
simulators treating complex spatial geometries, e.g. NEURON \cite{Hines_2001}.
More recently, due to interest in the computational properties of spiking 
neurons and their relevance to experimental observations, simulators targeting
specifically spiking neurons have been prominent, e.g. Brian 
\cite{Goodman_2009}.

However, another level description of neural dynamics has been treated
in the literature of neural mass models and neural fields 
\cite{Deco_2008a, Coombes_2010}. Here, the spatial
extent of the modeled dynamics is far larger and hence permits networks 
thereof to scale reasonably to the entire cortex, under the assumptions 
of the models, when combined with empirical measurements of cortico-
cortical connectivity. Therefore, the physical scale modeled by the TVB
simulators differs from that for which other simulators were designed.
Several technical issues stem from this scale, e.g. efficient handling
of dense $N^2$ delays and neural field-like connectivity, which will be
discussed in more detail below. 

%Other simulators compared to TVB

	Brian should be a particular focus in this section, as it may
    be one of the closest.  \note[lp]{Dana also (but is it alive?)
    (http://dana.loria.fr/index.html)}

Large scale simulation implies flexible integration. We shall see
how this is enable by the architecture..
\note[mw]{expand}

\subsection{Practical informations / contributors information}

\note[sk]{this para is repeats part of the first para of the framework section}
To address these concerns, a flexible architecture was developed to
allow easy integration of any computational tools along with a system
for describing typically types of data. A web based UI was developed
for users not comfortable with programming, as well as \matlab{} toolbox
for interacting with the Python based framework, given that many
neuroscientists are already comfortable with the \matlab{} workflow.

\note[sk]{this paras content is useful, but it seems out of place}
Lastly, a high performance, highly documented simulator along with
various forward solutions have been implemented and released under a
GPL licence to ensure universal access to high quality simulations, 
developed on the well-known Github, making it extremely easy for 
anyone to contribute.

TVB source code is available for download on Github at
\url{https://github.com/the-virtual-brain/}.  Previous Git and Python knowledge
is required for contributing.  Although you could independently install Python
and the rest of TVB dependencies on your machine, and then use the code from our Github repository
as a simple local clone, we recommend you download \emph{TVB\_ Distribution}
from \url{http://www.thevirtualbrain.org/register/}, fork our repositories on
Github and further use \emph{contributor\_setup} script, from inside \emph{TVB\_Distribution} 
folder, to link the two.  This recommended use-case will provide you with
all of TVB's dependencies via \emph{TVB\_Distribution}.
\note[lp]{Is there any plan for a .deb package with
full dependencies taken into account in this context? Or Pypi?} 

\note[lp]{Generic description and goal of the paper}

The overall structure of TVB is depicted in Figure~\ref{fig:architecture} where
components of the framework and of the scientific library are shown with
their relationships.

 \begin{figure*}
        \centering
        \includegraphics[width=0.90\textwidth]{images/architecture.jpg}
        \caption{TVB architecture: Yellow blocks are part of the Scientific
            Library of TVB, while the green blocks are part of TVB Framework.
            TVB provides two independent interfaces, depending on the
            interaction type wanted by the end-user (web or console).  TVB
            Storage layer is compulsory for the web interface, but it can be
            switched on/off for the console interface.  \note[lp]{It is said in
                the text that "console interface" is part of the "architecture"
            and not the "scientific library", this is the contrary on the
        figure}
        \note[lp]{What is a "S-User"? I missed the definition?}
         }
        \label{fig:architecture}
 \end{figure*}

 The goal of this article is firstly to describe TVB's architecture and 
 implementation from the
 development point of view and demonstrate how it interacts with other tools
 and how it can be extended (on the basis of extension already integrated in
 TVB).


\section{Architecture}
TVB is logically and technically divided into a scientific
library and a framework package, where the scientific library includes
datatypes, basic analyses and the simulator, while the
framework handles execution infrastructure, the web-based user interface and
data storage.  The scientific library can function as an independent Python
module, but the framework depends the scientific library for datatype definitions
and algorithms. 

\subsection{Basic Concepts}

The TVB framework is oriented around data and the operations that introduce,
generate, transform and visualize the data. The relevant interface classes
derive from the metaclasses in Python's abstract base class library, and
provide a foundation for defining any type of data from a tuple to a
set of EEG channel labels to a simulation, each of which is defined by a class
implemented the datatype interface and possessing one or more datatypes as 
attributes. An operation is any algorithm that
has been \emph{adapted} to the framework in a class implementing the abstract
adapter interface, and range from the simulator to data importers to
visualizers.  The goal of these twin, generic abstractions is to provide a
solid basis on which to implement a storage back-end, workflow management and a
number of features to support collaborative work. 

Importantly, the framework supports both the use of the web-based graphical
interface and the console interface for advanced user and developers. Where
the console user does not wish to rely on the database persistence, this 
storage layer can be disabled. Specifically, TVB uses the notion of \emph{profile} to 
identify in what context the application is currently running,
and thus what components are expected to be loaded.
For example, when the scientific library is used alone, a specific profile (\emph{library profile}) class 
gets linked as current profile, which, in this case, disables data storage and the web interface. Other profiles available
in TVB are: \emph{command profile}, \emph{deployment profile} (with web interface), and \emph{test profiles}.


%\begin{verbatim}
%    def configure(self, **kwargs):
%        To be implemented in each Adapter that requires any specific configurations
%        before the actual launch.
%
%    def get_input_tree(self):
%        Describes inputs and outputs of the launch method.
%
%    def get_output(self):
%        Describes inputs and outputs of the launch method.
%
%    def get_required_memory_size(self, **kwargs):
%        Abstract method to be implemented in each adapter. Should return the required memory
%        for launching the adapter.
%
%    def get_required_disk_size(self, **kwargs):
%        Abstract method to be implemented in each adapter. Should return the required memory
%        for launching the adapter in kilo-Bytes.
%
%    def get_execution_time_approximation(self, **kwargs):
%        Method should approximate based on input arguments, the time it will take for the operation 
%        to finish (in seconds).
%
%    def launch(self):
%         To be implemented in each Adapter.
%         Will contain the logic of the Adapter.
%         Any returned DataType will be stored in DB, by the Framework.
%\end{verbatim}


\subsection{Datatypes and storage}

\subsubsection{TVB Traits}

Because an explicit goal of TVB was to provide a user interface to each of the
entities and algorithms contained within, it is necessary at some point to
provide metadata on how to built that interface. A traits system was
developed, similar to that of IPython or EPD, allowing for
attributes on a TVB class to be written out with its full metadata. An extensive
set of building blocks are already implements from numeric types and arrays to
lists, tuples, string, and dictionaries.

\begin{center}
	\begin{table*}[ht]

	\begin{tabularx}{\textwidth}{lll}
      		\toprule
      		Traited Attribute    & Description  \\ 
      		\midrule
		default 	& Default value for current attribute. Will be set on any new instance if not specified otherwise in the constructor.  \\
		console\_default & Define how a default value can be computed for current attribute, when console interface is enabled. \\
		range	& Specify the set of accepted values for current attribute. Mark that this attribute is usable for parameter space exploration. \\

		label		& Short text to be displayed in UI, in front of current attribute. \\
		doc		& Longer description for current attribute. To be displayed in UI as help-text. \\
		required	& Mark current attribute as required for when building a new instance of the parent class. \\
		locked	& When present and \emph{True}, current attribute will be displayed as read-only in the web interface. \\

		options	& Used for attributes of type \emph{Enumerate}, specifying the accepted options as a list of strings. \\
		filters\_ui	& SQL filters on other attributes, to be applied in UI. \\
		select\_multiple & When \emph{True}, current attribute will be displayed as a select with multiple options in UI (default is single-select) \\
		order	& Optional number identifying the index at which current attribute will be displayed in UI. \\
				& When negative, the attribute is not displayed at all. Ascending order for indices is considered when displaying. \\

		use\_storage	& When \emph{False}, current attribute is not stored in database or file storage. \\
		file\_storage	& Valid values for this attribute are: \emph{None} , \emph{HDF5}, or  \emph{expandable\_HDF5}, \\
					& When \emph{None}, current attribute is not stored in the file-storage at all. When \emph{HDF5}, we use regular H5 file storage. \\
					& When \emph{expandable\_HDF5} value is set, a H5 stored in chunks is used. \\
		\bottomrule
    	\end{tabularx}
  	\caption{TVB currently available Traited Attributes}
  	\label{tab:traits}
	\end{table*}
\end{center}


When methods of such a class with annotated attributes are invoked, they may use
the traited attributes directly, accessing either a default value or one given
during the instantiation of the object. Additionally, this allows the web-based
user interface to introspect a class for all of its attributes and their
descriptions, to provide help and choose the proper display form. The explicit
typing also allows such classes to be nearly automatically mapped to storage
tables, providing persistence, when the storage layer is enabled.  Lastly,
because such metadata is used to build the docstring of a class, the console
user also may obtain extensive descriptions of class, attributes, methods and
arguments in the usual way. Table \ref{tab:traits} lists the various parts 
of a traited attribute and how they are used. 


\subsubsection{Datatypes}

In scientific Python code, it is conventional to provide arguments
of an algorithm as a ``bare'' array or collection there of, and sanity
checks of arguments proceed on the basis of array geometry, for example.
In TVB, we consider a \textit{DataType} to be a full, formal description of 
an entity involved in an algorithm that would be part of TVB. 

In TVB, datatypes represent the common language, to be used between different
application parts: like uploaders, analyzers, simulator and visualizers.
Some of the algorithms are producing these datatypes, while others are reading
them as input.  In order to decouple the definition and several usages of such
entities, datatypes are declared outside the algorithms and shared between them.
For example an instance of datatype TimeSeriesRegion is created by the
Simulator, and it can be accepted as input for several visualizers or analyzed
by principal component analysis and cross coherence algorithms.

Technically, TVB datatypes are annotated Python classes, which
contain one or more fields and associated descriptive information, as
well as methods for operating on the data they contain. The definition of a
datatype is achieved using TVB's traiting system, mentioned in previous section.

For example, the \texttt{Connectivity} datatype, which may elsewhere
be represented by a simple $N$ by $N$ NumPy array, is written as a class
in which one of the attributes, \texttt{weights}, is a explicitly typed 
\texttt{FloatArray}, and the declaration of this type is complemented by
explicit label, default values, and documentation strings. See
Code~\ref{lst:ConnectivityData}.

\begin{lstlisting}[caption={The COnnectivityData listing},
                   label={lst:ConnectivityData}]
class ConnectivityData(MappedType):

  region_labels = arrays.StringArray( 
	label="Region labels", 
        doc="""Labels for the regions ...""")

  weights = arrays.FloatArray( 
	label="Connection strengths",
        doc="""... strength of connections ...""")

  tract_lengths = arrays.FloatArray( 
	label="Tract lengths",
        doc="""... length of myelinated fibre tracts.""")

   speed = arrays.FloatArray( 
	label="Conduction speed", 
	default=numpy.array([3.0]), 
	file_storage=core.FILE_STORAGE_NONE,
         doc="""... matrix of conduction speeds ...""")

  centres = arrays.PositionArray( 
	label="Region centres",
        doc="""... locations for the region centers""")
\end{lstlisting}
	

\subsection{Adapters}

The framework expects algorithms to be adapted by providing a class
which inherits from the base adapter, \texttt{ABCAdapter}, implementing 
the adapter interface:

\begin{lstlisting}[caption={The ABCAdapter listing},
                   label={lst:ABCAdapter}]
class AdapterExample(ABCAdapter):

  @abstractmethod
  def get_input_tree(self):
  	pass

  def configure(self, **kwargs):
  	pass

  @abstractmethod
  def launch(self):
  	pass
\end{lstlisting}

\noindent where \texttt{get\_input\_tree} builds a dictionary of input
arguments required for the algorithm and for
presenting menus and fields in the user interface, \texttt{configure} allows the 
adapter to initialize itself and its algorithm based on arbitrary arguments
and \texttt{launch} invokes the algorithm.
Additional methods include \texttt{get\_output}, \texttt{get\_required\_memory\_size},
and
\texttt{get\_required\_disk\_size}.

Several categories of adapters have been defined in TVB: 

\begin{itemize}
	\item \textit{creators} which are internal algorithms for producing datatype instances. 
		Each creator has one or multiple pages in the web interface, in which the user
		 configures input parameters and chooses from the available options for computing a particular datatype.

	\item \textit{uploaders}: allow the upload into TVB framework of external data, 
    		such as \emph{gifti} files of plain \emph{csv} files.

	\item \textit{simulator} is an adapter for the simulator, adjusting it to fit
		the workflow mechanisms inside the framework .

	\item \textit{analyzers} which offer the interface to libraries containing algorithms 
		for the analysis of the data (wavelets, FastICA, BCT, etc).

	\item \textit{visualizers}, derived from the \emph{ABCDisplayer} base class, prepare a datatype
		for display. Each visualizer (Python adapter class) requires a complementary set
		of Javascript and HTML files. 

	\item \textit{portlets} provide a chain of analyzers leading to a visualizer.

	\item \textit{exporters} prepare a datatype for export \& download.

\end{itemize}

Note that the adapters and datatypes are intended to provide full 
power and flexibility of the framework; when the simulator is invoked from
the web-based UI, it is done so through a \texttt{SimulatorAdapter} which,
despite being relatively complex, is built with \emph{traits} all the way down.

It is reasonable to ask what such a scheme offers over the more 
conventional approach of Python, where presumably it would have been
sufficient that each adapter consist of a class with an \texttt{\_\_init\_\_}
and \texttt{\_\_call\_\_} method, in the case of a function type. 
We note that because in the case of TVB, the context in which an object
is used is more varied, e.g. not simply initialized but loaded through 
SQLAlchemy's object relational mapping, and that the adapter is required to perform more tasks
than just initialization and invocation, e.g. provide expected shape of 
result, estimate occupied memory and do not start if insufficient resources are found on current machine,
 it was advantageous to create a distinct set of interfaces built on top of
the abstract base class framework provided by Python's standard library.

\paragraph{An adapter for FastICA}

An example of an adapter applying the FastICA algorithm to each of the 
state variables in a simulation time series is given in Code \ref{lst:ica}.

\lstinputlisting[caption={ICA adapter for FastICA library},
                 label={lst:ica}]{ica_adapter.py}

\paragraph{Interfacing with MATLAB}

One of the well-known libraries for characterizing anatomical 
and functional connectivity is the \emph{Brain Connectivity Toolbox} 
\citep{Rubinov_2010}. 
Because it is written in MATLAB, with maintainers who prefer MATLAB, we 
chose not to port routines of the library to Python but instead build
a MATLAB adapter which runs arbitrary MATLAB code. 

This generic Matlab adapter works by generating at runtime a script with MATLAB code, 
wrapping the script call in Python with a try-except clause,  
loading and saving the workspace before and after the call,
generating a workspace \texttt{.mat} file, invoking the MATLAB or Octave
executable, and loading the resulting workspace file. 

Despite invocation of MATLAB being a relatively slow operation, this works
without problems in a single user situation, and where Octave is available, it
is quite fast. In the case that many operations are necessary, they can be
batched into the same run.


\section{Simulator}
The TVB simulator resembles popular neural network simulators in 
many fundamental ways, both mathematically and in terms of informatics 
structures, however we have found it necessary to introduce auxiliary
concepts particularly useful in the modeling of large scale brain 
networks. In the following, we will highlight some of the interesting
principles and capabilities of TVB's simulator and give rough characterization
of the execution time and memory required in typical simulations.

\subsection{Node dynamics}

	In TVB, nodes are not considered to be abstract neurons nor necessarily 
	small groups thereof, but rather large populations of neurons. Concretely, 
	the main assumption of the neural mass modeling approach in TVB is that
	large pools of neurons on the millimeter scale are strongly approximated
	by population level equations describing the major statistical modes of 
	neural dynamics. Often, averaging techniques are employed, though techniques
	retaining several modes have been developed \cite{Stefanescu_2011}.
	Such an approach is certainly not new; one of the early
	examples of this approach consist of the well known Wilson-Cowan equations
	\cite{Wilson_1973}. Nevertheless, there are important differences in the
	the assumptions and goals from modeling of individual neurons, where the
	goal may be to reproduce correct spike timing or predict the effect of 
	a specific neurotransmitter. A second difference lies in coupling:
	chemical coupling is often assumed to be pulsatile, or discrete, between neurons, whereas
	it is considered continuous.
	Typically the goal of neural mass modeling
	is to study the dynamics that emerge from the interaction of two
	or more neural masses and the network conditions required for stability
	of a particular spatiotemporal pattern. In the following, we shall 
	briefly discuss some of the models available in TVB.

	As we have noted, many neural mass models have been developed. One of
	the more prominent examples in the systems neuroscience literature is 
	that the Jansen-Rit model of rhythms and evoked responses arising from
	coupled cortical column \cite{Jansen_1995}. 
	Advantages of the Jansen-Rit model stem from the connection made
	between empirical studies of neural tissue and the model's parameters, 
	making it easier in certain cases to make concrete predictions about
	the relation between a dynamical regime and its neurobiological 
	mechanism. However, because the form of the model used often employs
	at least six dimensions, it is not always clear how to analyze or
	visualize. Lastly, the model requires frequent computation of exponentials,
	requiring considerable computational time. 

	For these reasons, it is often desirable to have a simpler mathematical 
	model, which may be reproduce the same qualitative phenomena as other 
	models, implemented with fewer and simpler equations. Such is the motivation
	for the generic two-dimensional oscillator model provided by TVB. 
	Model produces oscillations, damped, spike-like or 
	sinusoidal activations. While these alone are not interesting, they 
	permit the study of network phenomena, such as synchronization of rhythms
	or propagation of evoked potentials, while requiring less time to simulate.

	However, the modeler's goals may not lead to either the Jansen-Rit or 
	generic 2D oscillator, and several other mass models are provided by TVB:
	the previously mentioned Wilson-Cowan description of functional dynamics of
	neural tissue \cite{Wilson_1972}, the Kuramoto model of synchronization \cite{Kuramoto1975}, two and 
	three dimensional mode-level models describing populations with 
	excitability distributions \cite{Stefanescu_2011, Stefanescu_2008} are
	among the available models in TVB.  
	Again, should any of these be insufficient, a new model can be implemented
	with minimal effort by subclassing a base \texttt{Model} class and providing a 
	\texttt{dfun} method to compute the right hand sides of the differential 
	equations. Please refer to the \texttt{tvb.simulator.models} module for 
	examples.

	Because not all of the state variables of a model have the same meaning,
	for each of the models, we note two sets of variables, 
	coupling variables, which must be propagated by the connectivity to 
	provide input, and variables of interest whose values are recorded by
	monitors and saved for further analysis. 

\subsection{Network structure}

	The network of neural masses in TVB simulations directly follows from 
	a pair of 
	geometrical constraints on cortical dynamics. The first is the 
	large-scale white matter fibers that form a non-local and heterogeneous
	(translation variant) connectivity, either measured by anatomical
	tracing (CoCoMac\ref{CoCoMac}) or diffusion-weighted imaging. The second
	is that of horizontal projections along the surface, which are 
	usually modeled through a translationally invariant connectivity kernel
	though as with most parameters in TVB, spatially inhomogeneity
	is supported as well due to the use of generic NumPy operations which
	broadcast dimensions automatically. 

	\subsubsection{Large-scale connectivity}

	The large-scale region level connectivity at the scale of centimeters,
	resembles more a traditional
	neural network than a neural field in that neural space is discrete, 
	each node corresponding to a neuroanatomical region of interest, such
	as V1, etc. It is at this level that inter-regional delays play a large
	role. 

	It is often seen in the literature that the inter-node coupling functions
	\textit{are} part of the node model itself. In TVB, we have instead 
	chosen to factor such models into the intrinsic neural mass dynamics, where each 
	neural mass's equations specify how connectivity contributes to the
	node dynamics, and the coupling function, which specifies how the activity
	from each region is mapped through the connectivity matrix. Common coupling 
	functions are provided such as the linear, difference and periodic functions
	often used in the literature.

	\subsubsection{Local connectivity}

	The local connectivity of the cortex at the scale of millimeters provides
	a continuous 2D surface along horizontal projections connect 
	cortical columns. Such a structure has previously been modeled by
	neural fields \cite{Amari_1977,Jirsa_1997}. In TVB, a cortical mesh, 
	as obtained from structural MRI and simplified, provides a spatial 
	discretization on which neural masses are placed and connected with a
	connectivity kernel, itself only a function of the geodesic distance
	between the two masses, and this is considered to provide an
	adequate approximation of a neural field, depending on the properties
	of the mesh and the imaging modalities that sample the activity simulated
	on the mesh \cite{Spiegler_2013}. 
	\note[sk]{The implementation of the local connectivity kernel is such
	that is can be re-purposed as a discrete Laplace-Beltrami operator,
	allowing for the implementation of true neural-field models that 
	use a second-order spatial derivative as their explicit spatial term.}

	TVB currently provides several connectivity kernels, among others, 
	the Laplacian and Gaussian kernels. Once a cortical surface mesh 
	and connectivity kernel and its parameters are chosen, the geodesic
	distance (i.e. the distance along the cortical surface) is evaluated
	between all neural masses \cite{Mitchell1987}, and a cutoff is chosen
	past which the kernel falls to 0. This results in a sparse matrix that 
	is used during integration to implement the approximate neural field. 

\subsection{Integration of stochastic delay differential equations}

	In order to obtain numerical approximations of the network model 
	described above, TVB provides both deterministic and stochastic
	Euler and Heun integrators,
	following recent literature on numerical solutions to stochastic
	differential equations \cite{Kloeden_1995,Mannella_2002,Mannella_1989}.

	While the literature on numerical treatment of delayed or 
	stochastic systems exists, it is less well known how to treat 
	the presence of both. For the moment, the methods implemented by TVB
	treat stochastic integration separately from delays. 
	This separation conincides with a modeling assumption that in
	TVB the dynamical phenomena to be studied are largely determined
	by the interaction of the network structure and neural mass dynamics, 
	and that stochastic fluctuations do not fundamentally reorganize the
	solutions of the system \cite{Ghosh_2008,Deco_2009,Deco_2011,Deco_Senden_2012}.

	Due to such a separation, the implementation of delays in the
	regional coupling is performed outside the integration step,
	by indexing a circular buffer containing the recent simulation 
	history, and providing a matrix of delayed state data to the 
	network of neural masses. While the number of pairwise
	connections rises with $n_{region}^2$, where $n_{region}$ is
	the number of regions in the large-scale connectivity, 
	a single buffer is used, with a shape
	$(horizon, n_{cvar}, n_{region})$ where $horizon = max(delay) + 1$,
	and
	$n_{cvar}$ is the number of coupling variables. Such a scheme helps 
	lower the memory requirements of integrated the delay equations.

\subsection{Forward solutions}

	A primary goal of TVB is not only to model neural activity itself
	but just as importantly the imaging modalities common in human 
	neurosciences, using so-called forward solutions, which allow for
	the projection of neural activity into sensor space. To account
	parsimoniously for other ways in which simulated data might be saved, 
	such as simple temporal averaging, we refer to each of these simply as 
	\textit{Monitors}, which take as input neural activity and 
	output a particular projection thereof. In most cases, this 
	takes the discrete-time form of

	\[ \hat{y}[j, t] = \sum_{i=1, \tau=1}^{N_W, N_k} W[j, i] K[\tau] y[i, t-\tau] \]

	\noindent where $y[i, t]$ is the amplitude of the $i^{th}$ neural mass at time
	$t$, $K[\tau]$ is a temporal kernel, and $W[j, i]$ is a spatial kernel,
	usually projecting the state variable of interest of the $i^{th}$ 
	neural mass to the $j^{th}$ sensor. 

	Where necessary for computational reasons, monitors employ more than 
	one internal buffer. The fMRI monitor is one 
	example: given a typical sampling frequency of simulation may be upward of 
	64 kHz, and the haemodynamic response function may last several seconds, 
	requiring many gigabytes of memory for the fMRI monitor alone. Given that 
	the time-scale of simulation and fMRI differ by several orders of magnitude, 
	the subsequent averaging and downsampling is justified. 

	In the cases of the EEG and MEG monitors, $K$ implements a simple
	temporal average, and $W$ consists of a so-called lead-field matrix as typically
	derived from a combination of structural imaging data of the patient 
	and the locations and orientations of the neural sources and the locations
	and orientions of the EEG electrodes and MEG gradiometers and magnetometers. 
	As the development and implementation of such lead-fields is well developed
	elsewhere \cite{Jirsa_2002,Nolte2003,Gramfort_2010}, TVB provides access
	to the well-known OpenMEEG package, however, the user is free to provide 
	his or her own.

\subsection{Performance}

	A primary goal of the simulator is to be available as a pure Python package,
	and secondarily, to be fast enough. We have not found it useful to 
	develop theoretical estimates of the time and space complexity of the 
	algorithms, given that much of the heavy lifting is already done in native
	code by NumPy and other standard libraries. Instead, 
	in the following, we profile a set of eight characteristic simulations
	on both memory use, specifically the heap size as measured by Valgrind's 
	\texttt{massif} tool \cite{valgrind2007}, and function timing as measured by the 
	\texttt{cProfile} module of the standard library. 
	
	Measurements were
	performed on an HP Z420 workstation, with a single Xeon E5-1650
	six-core CPU running at 3.20 GHz, L1-3 cache sizes 384 KB, 1536 KB
	and 12 MB respectively, with main memory 4 x 4 GB DDR3 at 1600 Mhz,
	running Debian 7.0, with Linux kernel version 3.2.0-4-amd64. 
	The 64-bit Anaconda Python distribution was used with additional Accelerate
	pacakge which provides acceleration of common routines based on the 
	Intel Math Kernel Library. A Git checkout of the trunk branch of TVB 
	was used with SHA 6c644ab3b5.

	Eight different simulations were performed corresponding to the combinations of
	either the generic 2D oscillator or Jansen-Rit model, region-only
	or use of cortical surface, and two conduction speeds, $v_c = 2.0$ and
	$v_c = 20.0$ (m/s). In each case, a temporal average monitors at 512 Hz
	is used, and the results are discarded. The region-only simulation was
	run for a second while the surface simulation was run for 100 ms. 

	\note[mw]{Table of profiling results to go here. Profiling has been done, 
	curating results now..}


\section{User Interaction}
\subsection{Graphical Interface Interaction}

    A graphical web interface was chosen as solution to give users a quick
    interaction with \TVB  . The web interface is easy to access (local or
    remotely) through a web browser, it can be used by different types of
    users, including those without programming knowledge, and it offers
    great support while learning about \TVB concepts and workflow
    expectations.  In our architecture diagrams, the actor accessing \TVB
    through the web interface is called a \emph{G-User}.

    The http is served using \emph{Cherrypy}
    \texttt{http://www.cherrypy.org/} which is a minimalist, object-
    oriented web framework,  in combination with \emph{Genshi} templating
    system, to support the separation of layers as guided by \emph{MVC
    (Model View Controller)} pattern.

    \subsubsection{Projects, Accounts, Operations \& Data}

\TVB uses entities like: Account, Project, Operation, DataType and Workflow, for modeling G-User actions and artifacts. 

 \begin{figure}[!htbp]
    \centering
    \subfloat[][]{\includegraphics[width=0.47\textwidth]{images/ui_projects.png}}
    \\
    \subfloat[][]{\includegraphics[width=0.47\textwidth]{images/ui_project_graph.png}}
    \\
    \subfloat[][]{\includegraphics[width=0.47\textwidth]{images/ui_project_operations.png}}
    \\
    \subfloat[][]{\includegraphics[width=0.47\textwidth]{images/ui_project_datatype_details.png}}
    \caption{\TVB Data Organization
    (A) View all Projects
    (B) 2D graph display of Operations with their input and output DataTypes 
    (C) View all Operations in current project with their status, duration, results, etc
    (D) DataType details and further available operations for it. This menu becomes available after clicking a DataType result from several places in \TVB }
        \label{fig:project}
\end{figure}

    An \emph{Account} or \emph{User} is needed for accessing \TVB through
    the web interface.  When \TVB web interface is fired for the first
    time, the G-User is requested to set his or her username and a
    password for the first account which acts under an
    \emph{administrator} role. Later on, more users can \emph{register}
    for other accounts, that should be previoulsy validated by the admin
    account.

    A \emph{Project} in \TVB is a logical grouping entity, which  can be
    used in several ways by the end-user;  for example one could choose to
    create a project for each experiment in \TVB, while others might
    create projects for each subject of a group. Each project has a single
    \emph{User} (or \emph{Account}) as owner, but a project can be shared
    with multiple users to allow for collaborative reasearch.

    Any execution of an Adapter results into an \emph{Operation} in the
    context of a project. Multiple operations will be executed under the
    same project. For example we will have operations created for each
    execution of a simulation, each run of a Fourier analyzer, or launch
    of a Brain Visualizer. An operation changes status over time, from
    \emph{started} into \emph{canceled}, \emph{finished successfully} or
    \emph{finished with error}. One operation can have multiple input and
    output parameters and parameters can be scalars or DataTypes.

    A \emph{Workflow} in \TVB is a set of operations with they artifacts,
    and wraps around a simulation as leading component. A workflow can be
    seen as a default \emph{tag} placed by the system on Operations and
    DataTypes  which are logically connected, as resulting one after the
    other. Custom tags can also be added by the end-user both on DataTypes
    and Operations, for tracking entities inside a Project.

    \subsubsection{Simulator Interface}
    
    \note[ld]{Fill description for this}
    
    \subsubsection{Analysis \& Visualizers}

\TVB does not aim to compete at the analysis level with other tools in
Neuroscience,  highly specialized and with great history in data
analysis, like  FSL or SPM.  What we offer is a minimalist set or
algorithms to post-process your  simulated results (or even process
imported patient measured scans) inside \TVB, mainly for quick
validations.

We have created inside \TVB adapters for \emph{Fast ICA} from the
python library \emph{sklearn},  we've implemented a python version of
\emph{Fourier Spectral Analysis},  we have even wrapped the Matlab
library \emph{BCT} \url{https://sites.google.com/site/bctnet/}, and
others as analyzers.

For each of the DataTypes produced in \TVB, one or multiple
visualizers are available.   \TVB has couple of  visualizer types,
each developed with the technology providing better support on the
specific requirements for the visualization in course:

 \begin{figure}[!htbp]
    \subfloat[][]{\includegraphics[width=0.48\textwidth]{images/ui_view_brain.png}}
    \\
    \subfloat[][]{\includegraphics[width=0.48\textwidth]{images/ui_view_coherence.png}}
    \\
    \subfloat[][]{\includegraphics[width=0.48\textwidth]{images/ui_view_topo.png}}
    \\
    \subfloat[][]{\includegraphics[width=0.48\textwidth]{images/ui_view_pse.png}}
    \caption{\TVB visualizers: 
    (A) WebGL: 3D display of region level simulated signal, mapped on a brain cortical surface
    (B) SVG: Cross Coherence
    (C) MPLH5:  Topograhic view with Connectivity in/out strength measures
    (D) FLOT: Parameter Space Exploration results grid}
        \label{fig:viewers}
\end{figure}

\begin{enumerate}

    \item \emph{WebGL viewers:} are based on \emph{HTML 5 Canvas} element
    and the \emph{gl} context. These viewers offer 3D nice display,
    vectorial zoom support, user interaction with the scene (rotate,
    translate), quick response (even when thousands of vertices and edges
    are to be manipulated) and good resolution for the images exported.
    
    \item \emph{SVG viewers:} offer great selection, zoom and scaling
    effects and extraordinary quality for the exported artifacts, while
    having a relatively low number of elements to display on the page. We
    use such viewers for manipulating and displaying TimeSeries,
    Covariance or Cross Coherence DataType results.

    \item \emph{MPLH5 viewers:}  emph{Matplotlib} has an \emph{HTML 5}
    backend that we use for viewing some of \TVB DataTypes (like
    Fourier or Wavelet) url{https://code.google.com/p/mplh5canvas/}

    \item \emph{Other} simpler viewers in \TVB are using JIT
    \url{http://philogb.github.io/jit/} or FLOT
    \url{http://www.flotcharts.org/} - JS libraries. These are mainly
    2D graph displayers for some simple \TVB generated data.

\end{enumerate}


\subsubsection{Connectivity Tool}

    \emph{Connectivity} in the context of \TVB is a DataType, mapping structural
    information about a subject (a real single individual or an average template). For
    editing and viewing a Connectivity, \TVB has a specific page, where
    the \emph{G-User} can manipulate connectivity strength and lengths
    starting from the granularity of an edge.

    We do not store or use information about the exact anatomical path or
    a connection, only the region centers and connection weights and
    lengths.

 \begin{figure}[!htbp]
    \centering
    \subfloat[][]{\includegraphics[width=0.48\textwidth]{images/ui_connectivity.png}}
    \\
    \subfloat[][]{\includegraphics[width=0.48\textwidth]{images/ui_connectivity_delays.png}}
    \caption{Connectivity Tools: 
    (A) Left side: Displaying weighted connections between selection of nodes, with 3D manipulation.
    Right side: Editing weight connections, singular or bulk.
    (B) Left side: Show effect of connectivity delays (in milliseconds) when conductions speed is 1 mm/ms.
    Right side: Editing and displaying one quadrant from the matrix of connection tracts.}
        \label{fig:connectivity}
\end{figure}



\subsection{Python Scripting}

\texttt{hello\_brain.py}
To give a basic feel for scripting \TVB simulations, we will 
walk through a simple example of a region-level simulation. We 
start with

\begin{lstlisting}
from tvb.simulator.lab import *
\end{lstlisting}

\noindent which is an all-in-one module making writing scripts
shorter, in the style of \texttt{pylab}, as it imports everything
from \texttt{pylab}, \texttt{numpy} and most of \TVB's simulator
modules. Next, we build a simulator object:

\begin{lstlisting}
sim = simulator.Simulator(
    model        = models.Generic2dOscillator(), 
    connectivity = connectivity.Connectivity(),
    coupling     = coupling.Linear(a=1e-2),
    integrator   = integrators.HeunDeterministic(),
    monitors     = (
        monitors.TemporalAverage(), 
    )
)
\end{lstlisting}

\noindent where we've employed a two dimensional oscillator
with default parameters, the default connectivity, a linear 
coupling function with a slope of $1e-2$, and deterministic
Heun integrator and a monitor that temporally averages the 
network dynamics before providing output.

While \TVB strives to keep modules independent of one another,
it is typical for mathematical dependencies to arise between, 
for example, the mass model and the integration time step, so
after configuring a simulator object, it is necessary to invoke

\begin{lstlisting}
sim.configure()
\end{lstlisting}

which results in walking the tree of objects, checking and 
configuring the constraints among parameters recursively.

The next step is to run through the simulation, collecting
output from the simulator. In this case, it is as simple as
\begin{lstlisting}
ys = array([y for ((t, y),) 
          in  sim(simulation_length=3e2)])
\end{lstlisting}
\noindent where the simulator has been called, returning a 
generator which performs the integration and returns, for each
monitor, the current time and activity. In a case where EEG 
and fMRI monitors, for example, were used, we might write
\begin{lstlisting}
eeg, mri = [], []
for (t_eeg, y_eeg), (t_mri, y_mri) in sim(3e2):
    if y_eeg is not None:
    eeg.append(y_eeg)
    ...
\end{lstlisting}
\noindent Because fMRI and EEG monitors have very different
timescales, whenever one monitor return data and the others do
not, the others contain \texttt{None}, hence the check. Building
more complex logic in this loop would permit, for example, online
feedback and modification of connectivity. 

After the simulation loop has finished, you may wish to see the
result, following the previous listing, 
\begin{lstlisting}
plot(ys[:, 0, :, 0], 'k', alpha=0.1)
\end{lstlisting}
\noindent Here we note that \texttt{ys} is four dimensional. The 
simulator has the convention of treating  mass model state as a
three dimensional array of state variables by nodes by statistical
modes. Because \texttt{ys} is an array collected over time, the first
dimension is time, and the plot here is of each node's first state
variable, over time.

Many more demonstrations of the various features of the simulator
can been found in scripts distributed with the sources of \TVB, or 
browsed online at \url{https://github.com/the-virtual-brain/scientific_library/tree/trunk/tvb/simulator/demos}.
In the next section, we will go into detail about the different
components of the simulator.

\subsection{MATLAB Scripting}

Due to the popularity of MATLAB in the neuroscience community, an
interface from MATLAB to \TVB has been introduced that allows a MATLAB
script to design a TVB simulation, run it on TVB and retrieve the 
results. The MATLAB toolbox is provided separately from TVB, at
\url{https://github.com/the-virtual-brain/matlab-tvb}.
In the following, we give a short demonstration and 
describe implementation and rationale.

Because the MATLAB functions need to know the address of the server,
so we take any of the URLs used by the Web UI (here, the one provided
when launching TVB):

\begin{lstlisting}
sv = vb_url('http://127.0.0.1:8080/user/')
\end{lstlisting}

To run simulations without blocking MATLAB, a multiprocessing Pool
is used. We reset the pool and change the number of processes to 6

\begin{lstlisting}
vb_reset(sv, 6)
\end{lstlisting}

Next, we can query the server for information on the classes available,
and also get help for each of the classes

\begin{lstlisting}
info = vb_dir(sv);
\end{lstlisting}

\noindent where info is a cell array of structs, one per module (models,
monitors, etc.) and each struct has a field per class (models.JansenRit, 
models.Kuramoto, etc.). Each of these fields contains the details on 
the class, including all the parameters that can be set. 

To build a simulation, we start with an empty struct
and fill in the details for each part

\begin{lstlisting}
sim = [];

sim.tf = 1e3 % simulation length milliseconds
sim.model.class = vb.models.Generic2dOscillator;
sim.model.a = -2.1;

sim.connectivity.class = 'Connectivity';
sim.connectivity.speed = 4.0;

sim.coupling.class = 'Linear';
sim.coupling.a = 0.002;

sim.integrator.class = 'HeunDeterministic';
sim.integrator.dt = 1e-2;
\end{lstlisting}

Monitors are specified similarly but as a cell array there may be
several of them:

\begin{lstlisting}
sim.monitors{1}.class = 'TemporalAverage';

sim.monitors{2}.class = 'Raw';
sim.monitors{2}.period = 1.0; % ms
\end{lstlisting}

\note[sk]{Raw monitor has no period, or rather the period can't be set as it is
fixed as the integration time step...}

Lastly, we submit the struct as a new simulation

\begin{lstlisting}
[id, data] = vb_new(sv, sim);
\end{lstlisting}

\noindent Lastly, results are returned in a struct, here named data
where each field contians the output of a monitor and can be plotted
and analyzed as a regular MATLAB dataset:

\begin{lstlisting}
plot(data.mon_0_TemporalAverage.ts,...
     squeeze(data.mon_0_TemporalAverage.ys)')
\end{lstlisting}

The implementation of this interface is a combination of an additional
CherryPy controller providing an HTTP/JSON API, running on the same 
server as the Web UI, and a set of MATLAB functions that send HTTP 
GET requests to the server. An implementation based on MEX functions 
invoking the Python library directly was considered, for reasons of 
performance, however, it was judged that such an implementation may be
difficult to stabilize and maintain, given that it would require binary
compatibility between MATLAB, Python and the C compiler. Two additional 
advantages of an HTTP API are that most computational environments have
the ability to connect and make HTTP requests, allowing other programs 
like Perl or Mathematica take advantage of TVB and the approach naturally
extends to work over the network, should TVB be running on another machine.


\subsection{Documentation}
\note[psl] Not sure if it's the best place to place this subsection 

\paragraph{Manuals}
\pargraph{Tutorials}






\section{Conclusions}

Since the recent release of the $1.0$ version of TVB, it has been 
officially considered \textit{feature} complete, however, in several
cases, the development of features has outstripped other essential 
parts of software projects. Going forward, general priorities include
advancing test coverage, improving documentation for users, and
preparing PyPI and Debian packages. In the mean
time, TVB's Google groups mailing list continues to fill any gaps. 

In the simulator itself, continued optimization of C and GPU code generation
will take place to increase the rate at which parameter sweeps can be
performed. Additionally, an interface \textit{from} MATLAB to TVB 
is being developed to allow use of the simulator through a simple
set of MATLAB functions. As this infrastructure is based on an HTTP and 
JSON API, it will likely enable other applications to work with TVB as well.

Lastly, as TVB was originally motivated to allow a user to move from
acquired data to simulated data as easily as possible, we will continue
to integrate several of the requisite steps that are not currently 
covered, such as analysis of DSI data to produce connectivity matrices,
though in many cases, such as parcellation and labeling of 
cortical areas, these steps may continue to require 
intervention with other software. Nevertheless, TVB provides a promising 
platform for integrating 
neuroinformatics tools with an emphasis on analysis and modeling of 
human neuroimaging data.

%\begin{enumerate}
%	\item Diffusion tensor imaging \& tractography pipeline to convert
%        MRI data into connectivity data.
%	\item Connectome project on structural connectivity
%	\item Structural imaging processing via FreeSurfer (pySurfer, NiPy, etc.)
%	\item NeuroML model specification 
%\end{enumerate}



\section*{Acknowledgments}
Several authors have also participated in the
development of TVB. They are listed in the \texttt{AUTHORS} file 
of the source code and deserve also our warm acknowlegments. LP whishes to thank
specifically Y.~Manhoun for his implication in the conception and development
of the first prototype of the TVB architecture. The research reported herein
was supported by the  Brain Network Recovery Group through the James S.
McDonnell Foundation and the FP7-ICT BrainScales. PSL and MMW received
support by from the French Minist\`{e}re de Recherche and the Fondation
de Recherche Medicale.

\bibliographystyle{apalike}
\bibliography{bib}

\end{document}

